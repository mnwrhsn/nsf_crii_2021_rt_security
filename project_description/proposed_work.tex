\section{Proposed Work: Multicore Security Integration} \label{sec:proposed_work}

\subsection{Reactive Security Mechanisms for Multicore Systems}

Recall that Contego~\cite{mhasan_ecrts17} is developed for single core systems only and multicore security integration frameworks (\eg Hydra~\cite{mhasan_date18, mhasan_date20}) do not support runtime mode changes upon suspecting security breaches. Integrating reactive security mechanisms for legacy multicore platforms (where designers have less flexibility for changing system architecture/parameter) is still an open problem.

\subsection{Atomiticity and Task Dependency}

Once we have developed the above reactive security mechanism for multicore RTS, we will extend our framework to \ca allow atomic (\ie non-preemptive) operations and \cb support inter-dependent execution.
Security tasks so far have been treated as being independent and preemptible. Depending on the operation, some security tasks may need to be executed \textit{without preemption} (atomic operation). As an example, consider a security task that scans the process table and has been preempted in the middle of its operation. An adversary may corrupt the process table entry that has already been scanned before the next scheduling point of the security task. When the security task is rescheduled, it will start scanning from its last known state and may not be able to detect the changes in a timely manner. One way to support such case is the following: when any security task needs to perform special atomic operations, the priority of the tasks can be increased to one that is strictly higher than all (or some) of the real-time tasks, depending on the requirements of the security event. We note that such atomic operations (using priority inversions) may compromising the timing constraints of some (or all!) of the real-time tasks. Hence, schedulability analysis needs to consider this. Besides,The scheduling policy should identify which real-time or security tasks can be dropped (or perhaps reschedule to other cores) to provide better trade-off between real-time performance and security defense.

 Further, security tasks may have dependencies where one task depends on the output from one or more other tasks (\ie they may need to follow certain \textit{precedence constraints}). For example, an anomaly detection task may depend on the outputs of multiple scanning tasks. 
 %Or, the scheduling framework may need to follow certain \textit{precedence constraints} for security tasks. For example, in order to ensure integrity of monitoring security, the security tasks' own binary may need to be examined first before checking the system binary files. 
 In such cases we may not independently execute the security tasks in parallel into multiple cores. To address this, 
 %we will consider the problem of integrating security tasks with dependencies between them. In particular, 
 %we proposed to
 we will use a directed acyclic graph (DAG) to capture the dependencies and constraints among security tasks and explore scheduling techniques for DAG-based tasksets. %tasksets to integrate such tasks into real-time systems. 
 In this case, cumulative tightness of achievable periodic monitoring proposed in Section X may no longer be a reasonable metric. We will include the constraints to ensure that the entire DAG is executed often and study what metrics can better capture security dependency requirements. \todo{fine-tune!}
 % and evaluate with different metrics that captures such dependecny re.

\subsection{Extension for Dynamic Priority Systems}

Our preliminary work is designed for fixed-priority schedulers since they are commonly used in commercial systems. However there exist dynamic scheduling algorithms (\eg EDF and LL?) that provide higher schedulability bounds.  Such dynamic scheduling algorithms can potentially increase the capacities of our proposed security integration frameworks  and eventually provide better tightness of monitoring for the security tasks. Integrating security mechanisms using dynamic scheduling algorithm such as EDF~\cite{?}, however, requires further analysis due to run-time priority changes of the real-time and the security tasks. In particular, the opportunistic sporadic security task model may need to updated with an objective of reducing the average response time and provide better tightness of periodic monitoring without compromising the schedulability of real-time tasks. We also intend to instrumenting those mechanisms in a commodity real-time OS with dynamic scheduler (say \texttt{SCHED\_DEADLINE} in real-time Linux kernel~\cite{?}) and investigate that whether the security mechanisms are operated as those are intended to.Besides, it requires further investigation to determine whether this improved system utilization (at the cost of analysis/implementation overhead) can provide us additional security benefits.
 

\subsection{Security Metrics and Performance Evaluation}

\todo{check Sibin CAREER Page 12/13}

\subsection{Bringing it Together: Secure Hardware-Software Integration}



\vspace*{1.0em}
\begin{mdframed}[backgroundcolor=gray!10]
\vspace{-1.2em}
\hfill{}\fcolorbox{black}{gray!30}{\bf Research Tasks}\hfill{}

\begin{itemize}[leftmargin=*]
	\parskip 0pt
	\itemsep 0pt
	\item \rtask Blah
	\item \rtask Blah
	\item \rtask Blah
	\item \rtask Blah
	\item \rtask Blah
\end{itemize}
\end{mdframed}