\section{Evaluation} \label{sec:eval}

We will evaluate our proposed schemes using a variety of methods as we present below. 
%The evaluation process will demonstrate the feasibility of our research results. 
Our implementation details and source code will be \textit{\ul{publicly available}}.



\paragraph{Simulation-based Evaluation.} All of our approaches will first be evaluated using an \textit{internally developed simulator}. We will use the simulator 
%to analyze the scalability of the algorithms and also 
to test with a more diverse set of real-time task parameters to explore the design-space and scalability of the solutions/algorithms that have been developed. In the past, we extensively use open-source and custom in-house simulators~\cite{sdn_qos_rtss17, sdn_qos_infocom21, sdn_qos_secsdn20, mhasan_rtss16, mhasan_certs16, mhasan_date18, mhasan_twc14_1, mhasan_twc14_2, mhasan_tcom15_1, mhasan_tcom15_2} primarily to test the theories, and we will use similar techniques to evaluate our proposed schemes. 

%\textbf{Note:} It is common practice in the real-time systems community to generate a large number (tens of thousands) of synthetic task sets to explore the design space of the solutions/algorithms that have been developed.




\paragraph{System Integration.} 


\paragraph{Benchmarks.} 


\paragraph{Demonstrative Platforms.} 


