\section{Background, Adversary Model, and Related Research} \label{sec:background}

\paragraph{Real-Time Systems.}

\paragraph{System and Adversary Model.}

%\subsection{State-of-the-art}

\paragraph{Related Work.}

Researchers proposed architectural frameworks~\cite{securecore, mohan_s3a, slack_cornell} that use hardware/software mechanisms to
protect against security vulnerabilities. However, those frameworks require custom hardware-support for monitoring, and hence not suitable for legacy or off-the-shelf systems. Integrating monitoring and detection tasks for multicore RTS without custom hardware support is an open research problem. 



%\paragraph{Other Related Research.}

\section{Integrating Monitoring and Detection}

The focus of this proposed research is on integrating or retrofitting security mechanisms into RTS.  When integrating any security mechanisms into RTS, the designers need to ensure that they do not perturb or impact the real-time functions in any significant way while at the same time provide the necessary level of security. This is especially acute in the case of \textit{legacy systems} (\ie existing RTS where modification or perturbation of task parameters such as run-times, periods, and task execution orders is not always feasible.) Any security mechanisms that are introduced not only have to co-exist with existing real-time tasks without violating their real-time and safety constraints but also the {\em parameters of such legacy tasks cannot be adjusted to accommodate the security tasks.}  Not only must the security mechanisms work effectively but they must also not interfere with the deadlines of real-time tasks. For instance, any monitoring and detection mechanism has to be designed so that an adversary cannot easily evade it. This may require that the monitoring and detection tasks be run \textit{frequently}. However, the stringent timing constraints in hard RTS introduce additional complexities for the implementation of such cyber-security mechanisms. For instance, the strict deadlines for the completion of periodic real-time tasks may not allow for frequent execution of security mechanisms. Further, unlike in conventional computing systems, it may not be possible to execute the security mechanisms for arbitrary lengths of time. The challenge is then, how do we develop solutions that address the apparent tension between \textit{security requirements} (\ie having enough cycles for effective monitoring detection) and the \textit{timing and safety requirements} (\ie not interfere with deadlines)?

Hence, we propose to improve the security posture of RTS through integration of \textit{\textbf{``security tasks''}} (\ie tasks that are specific for intrusion monitoring and detection tasks purposes) while ensuring that the existing real-time tasks are \textit{not affected} by such integration. Security tasks could include protection, detection or response mechanisms, depending on the system requirements --- for instance, a sensor correlation task (to detect sensor manipulation) or an anomaly detection task (that checks possible intrusions)~\cite{caml_song2016}. Table~\ref{table:rtos} presents some examples of security tasks that can be integrated into legacy systems (\textbf{Note:} this is by no stretch meant to be an exhaustive list). Unlike single core systems, integrating security into multicore platforms is more challenging since designers have multiple choices across cores to retrofit security mechanisms. For instance,is it better to \textit{dedicate a core} to all the security tasks or is it better to \textit{spread them out} (in conjunction with the real-time tasks) and if so, to \textit{which cores}? It is not trivial to determine the execution frequency and core assignment of security mechanisms (\ie \textit{what} security tasks will execute on \textit{which core} and with \textit{what frequency}). This in itself a research challenge that needs to be investigated. However, to begin with, we propose the following two performance criteria:


\begin{enumerate}[leftmargin=0.2in]
	\item \textit{Monitoring Frequency:} In order to provide the best protection, the security tasks need to be executed quite often. On the one hand, if the interval between consecutive monitoring events is too large, the adversary may harm the system (and remain undetected) between two invocations of the security task. On the other hand, if the security tasks are executed very frequently then it may impact the schedulability of the real-time tasks. Herein lies an important trade-off between monitoring frequency and schedulability. 
	
	\item \textit{Responsiveness:} In some circumstances, a security task may need to execute with less interference from higher-priority tasks. For instance, consider the scenario where a security breach is suspected. In such an event the security task may be required to \textit{perform more fine-grained checking instead of waiting for its next periodic slot}. This may result in delayed execution of low-priority, non-critical, real-time tasks.
	However, the scheduling policy needs to ensure that the system remains secure without violating real-time constraints for critical, high-priority, real-time tasks. 
	
	%	\item \textit{Atomicity:} Depending on the operation, some of the security tasks may need to be executed \textit{without preemption}. For instance, let us consider a security task that scans the process table and has been preempted in the middle of its operation. An adversary may corrupt the process table entry that has already been scanned before the next scheduling point of the security task. When the security tasks are rescheduled, it will start scanning from its last known state and may not be able to detect the changes in a timely manner.  
\end{enumerate}




\paragraph{Preliminary Work: Contego~\cite{mhasan_rtss16, mhasan_ecrts17} and Hydra~\cite{mhasan_date18, mhasan_date20}.}

The PI's dissertation work~\cite{mhasan_rtss16, mhasan_ecrts17, mhasan_date18, mhasan_date20} provides the foundation of this research. 


