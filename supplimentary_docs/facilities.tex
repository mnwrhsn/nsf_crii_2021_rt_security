


\section*{FACILITIES}

%\paragraph{Department of Electrical Engineering and Computer Science.}

The PI is a faculty member in the department of Electrical Engineering and Computer Science (EECS). The proposed project leverages several computational infrastructures already in place at Wichita State University (WSU). The PI will utilize existing facilities, equipment, and resources to execute the proposed research successfully.

\vspace{0.5em}
\paragraph{Cyber-Physical Systems Security Research Lab.} 

The PI directs the Cyber-Physical Systems Security Research Lab (CPS2RL). This 800 square feet lab is located on WSU's main campus in Wallace Hall, Room 331. The proposed research activities will be conducted in this lab. The lab contains various evaluation platforms, including multiple embedded development boards (ZedBoard, Raspberry Pi, BeagleBone Black, and UP Extreme) running Linux operating systems, one FischerTechnik manufacturing testbed, and four 3D printers. The lab also includes four general-purpose workstations (2.4 GHz CPU, 8 GB RAM, and 256 GB solid-state hard drive) and necessary research spaces for graduate and undergraduate students. The students will use these development boards and workstations for simulations, prototype development, and experiments. The lab is available 24/7 for the PI and students.


\vspace{0.5em}
\paragraph{Office Spaces and Administrative Supports.} The PI has 100 square feet of private office space on WSU's main campus in Jabara Hall, Room 244. The office is appropriate for day-to-day activities and holding meetings for the project. The graduate and undergraduate students will have access to student cubicles in the PI's research lab (CPS2RL). The graduate students will also have access to student cubicles in Wallace Hall, Room 309. These offices are appropriate for the students' day-to-day scholarly activities required for the project. The PI and students also have secretarial and technical support services provided through their department.

\vspace{0.5em}
\paragraph{Outreach Activities.}
The PI will leverage WSU's existing outreach programs (\eg Shocker Engineering Academy, Research Experiences for Undergraduates program, and Security Education Hub) and work on broadening participation in computing (refer to the attached support letters). These resources are not included in the budget.



\vspace*{1em}
\section*{MAJOR EQUIPMENT}

Not applicable for this project.


\vspace*{1em}
\section*{OTHER RESOURCES}

\paragraph{Startup Support.} The PI has \$50,000 startup support. The PI will use \$3000 from his startup funds to purchase additional hardware (ground rover and robotic arm testbed) required for this project.

\vspace{0.5em}
\paragraph{General Resources.} The Office of research at WSU promotes the expertise of the faculty by facilitating all aspects of externally funded grants and contracts. 
%They provide oversight, assistance, and guidance to faculty for proposal preparation and the administration of awarded programs.
They will provide oversight, assistance, and guidance to administer this project once awarded.

\vspace{0.5em}
\paragraph{Other Computing/IT Resources.} 

WSU's high-performance computing cluster (BeoShock) has two large GPUs and 800 CPU cores. The cluster is available to the PI and students for large-scale simulations. All IT computing facilities are connected to the gigabit campus backbone. The PI and students have access to the computing facilities via Ethernet/Wireless connections. WSU also provides web spaces and dedicated URLs for hosting webpages. The university also provides video conferencing software (Zoom and Teams) that can be used to hold virtual meetings.