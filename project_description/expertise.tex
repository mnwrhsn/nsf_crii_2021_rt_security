


\subsection{Justification for Funding Request}


\paragraph{Foundations for Long-Term Research.} 

The activities proposed here are critical steps for the PI to launch his research career. The proposed work will provide solid foundations for developing security techniques in diverse domains --- from RTS to broader cyber-physical, IoT/edge systems, and even general-purpose computing platforms. While the immediate focus of this research is on integrating security monitoring mechanisms into multicore RTS, the PI believes that the ideas developed in this work can be extended to more general-purpose computing systems and will serve as the basis for future competitive research proposals (\eg NSF CAREER program). Once the proposed ideas have been conceptualized, in Year 3--5, the PI will study the security of hardware/software-based architectures such as real-time hypervisors~\cite{rtzvisor,rt_xen} and TrustZone-enabled RTS~\cite{mukherjee2019optimized,mhasan_iotsnp19}. In Year 5--7, the PI intends to study security/privacy issues of distributed cyber-physical systems such as UAV swarms~\cite{chmaj2015distributed} and vehicular networks~\cite{mhasan_v2x_survey_20}. %software-defined real-time networks~\cite{sdn_qos_rtss17, sdn_qos_infocom21}, and sensor-actuator networks~\cite{van1993sensor,lu2015real}. 
From Year 7 onward, the PI intends to investigate security and resiliency aspects of more general-purpose systems along with IoT/edge-style cyber-physical computing platforms and study emerging technologies such as smart manufacturing, autonomous/electronic vehicles, and robot-aided automation (especially for elderly/disable people). \textbf{Note:} we present the plans beyond Year 2 to show the potentials of our proposed research agenda and the PI's long-term goals.



%\paragraph{Nourishing Collaborative Efforts.}

\paragraph{Strengthening Collaborative Efforts.}

The PI is developing new collaborations with other researchers for interdisciplinary work. In his first nine months as a faculty, the PI has already initiated collaborations with Dr. Sergio Salinas (Wichita State), Dr. Gedare Bloom (U. of Colorado at Colorado Springs), and Dr. Shubhra Kanti (Auburn) on topics related to security and privacy of cyber-physical systems and human-robot interaction. We anticipate leading multiple joint NSF proposals in the future. These projects will not only solve complex, practical problems and  contribute to the growth and development of computing research but also allow the PI to establish a successful academic career.
%academic career. %and lead to joint NSF proposals in the future.

During his PhD, the PI has successfully collaborated with industrial research labs (SRI International and Toyota Motors) that result in multiple publications~\cite{mhasan_sri_17,mhasan_v2x_survey_20} and a patent~\cite{mhasan_sri_patent_1}. The PI intends to strengthen his ties with industrial research and contribute to improving the security of consumer products. Hence, there is a high likelihood that the PI's future research outcomes will be deployed and enhance the security and resiliency of critical cyber-physical systems.


\paragraph{Lack of Support for Seeding Future Research.} 

The PI devotes his career to building secure, trustworthy, and resilient cyber-physical computing platforms. An integral part of the PI's career goals is inspiring, educating, and mentoring the broader community including K-12, undergraduate, and graduate students, and fostering equity and diversity in STEM education. 
%The PI has successful track record of publicly releasing research implementations 
%(refer to the PI's GitHub repositories~\cite{mh_github}) 
% 
The PI intends to promote reproducibility in systems research and disseminate his scientific findings through open-source initiatives. 
 \textbf{The PI \ul{does not have sufficient funds}\footnote{The PI's startup funding supports only one graduate student for 24 months.} to support a doctoral student and initiate these research plans.  The funding support from NSF for the proposed research initiation activities will be one of the significant steps towards achieving the PI's long-term career goals.}

%\todo{talk about collaboration}

%\paragraph{Seeding Career Goals.}




%\subsection{PI's Expertise and Likelihood of Success}

\subsection{PI's Expertise}



The PI, with his diverse research background, is in a unique position to carry out the proposed
research agendas. The PI's research expertise includes \ca real-time and cyber-physical systems security~\cite{mhasan_rtss16,
	mhasan_ecrts17, mhasan_sri_17, mhasan_date18, mhasan_date20, mhasan_resecure_iccps,mhasan_resecure_iot,mhasan_iotsnp19}, \cb development of resilient cyber-physical networks~\cite{sdn_qos_rtss17,sdn_qos_secsdn20,sdn_qos_infocom21}, and \cc resource management in cellular wireless networks~\cite{mhasan_bc15,mhasan_tcom15_1,mhasan_tcom15_2,mhasan_twc14_2,mhasan_twc14_1}. 
%He brings extensive expertise in the design and analysis of resilient real-time cyber-physical systems. 
In recent years, the PI has been at the forefront of the research in real-time security and resiliency with contributions ranging from developing theoretical models~\cite{sdn_qos_infocom21,sdn_qos_rtss17,sdn_qos_secsdn20,mhasan_rtss16,
	mhasan_ecrts17, mhasan_date18, mhasan_date20} as well as designing system architectures~\cite{mhasan_resecure_iccps,mhasan_resecure_iot,mhasan_iotsnp19}. 
	His work on integrating security as a first-class principle of real-time schedulers~\cite{mhasan_rtss16} has won
the \textit{outstanding paper} and \textit{best student paper} awards at the IEEE RTSS. %\footnote{The premier conference in the field of real-time and embedded systems.} 
%The PI  brings extensive expertise in the design and analysis of resilient real-time cyber-physical systems. 
The PI has published his solutions in top real-time and networking conferences/journals including RTSS~\cite{mhasan_rtss16, sdn_qos_rtss17}, ECRTS~\cite{mhasan_ecrts17}, ICCPS~\cite{mhasan_resecure_iccps}, INFOCOM~\cite{sdn_qos_infocom21}, IoT~\cite{mhasan_resecure_iot}, TWC~\cite{mhasan_twc14_1,mhasan_twc14_2}, and Network~\cite{mhasan_network15}. 
%As detailed in Section~\ref{sec:reorder}, the PI has also recently obtained preliminary results, in particular, on developing randomization techniques for single-core real-time systems. 
%The PI also has access to necessary resources (\eg undergraduate/graduate students, hardware, testbed, and research facilities) at Wichita State University to carry out the proposed research agenda (Section~\ref{sec:eval}). 
The \textit{PI has all the resources} (\eg research computers, experimental platforms, and lab facilities) needed to complete this project successfully. 
The PI also has experience working with \textit{underrepresented communities} in computing, most notably \textit{women students} at all levels ---  doctoral, masters, undergraduate, and even middle school students~\cite{mhasan_middleschool_news}. Further, the PI will leverage Wichita State's existing outreach programs 
%\footnote{Supporting letters are attached.} 
(\eg Shocker Engineering Academy~\cite{wsu_sea} and Hub for Cybersecurity Education and Awareness~\cite{wsu_hcea}) as well as resources from other organizations (\eg CRA's undergraduate mentoring programs~\cite{cra_dreu} and NSA's K-12 initiatives~\cite{k12_cybertalk}) and work on broadening participation of high school students and minority communities in engineering.  Hence, the PI is ideal for developing real-time security techniques proposed in this project and fostering diversity in STEM education.


