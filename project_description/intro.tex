\section{Introduction} \label{sec:intro}

Cyber-physical systems such as avionics, nuclear power plants, automobiles, space vehicles, power generation and distribution systems, medical devices, industrial robots have been in existence for decades. 
Such systems, which often have \textit{safety-critical} properties and \textit{real-time} (\ie stringent timing) requirements. Any problems in real-time systems (RTS) could result in significant harm to humans, the system or even the environment.  
Systems with real-time requirements need to function correctly, but \textit{within their predefined timing constraints}, often termed as ``deadlines''.  
For example, consider real-time applications with stringent timing constraints such as deployment of a car's airbag or an industrial robot operates on a manufacturing conveyor line. 
A typical window for an airbag deployment (time between detection of collision and final airbag operations) is around $50$--$60$ ms~\cite{hussain2006vehicle} and the response time for robot movements (\ie placing and moving objects on the conveyor) is around $50$--$100$~ms~\cite{castelli2019development}. 

\begin{wrapfigure}{r}{0.295\textwidth}
%\begin{figure}[!h]
	\centering
	\includegraphics[scale=.35]{rt_deadline.pdf}
	\caption{Timeliness requirements of RTS.} 
	\label{fig:rt_deadline}
%\end{figure}
\end{wrapfigure}
If real-time applications failed to comply with their timing requirements (deadlines), the usefulness of results produced by the system drops sharply (see Fig.~\ref{fig:rt_deadline}). From the earlier airbag deployment and manufacturing robot example, if the application tasks fail to deploy the airbag in time or there is a delay to update the angle of rotation of the robot arm, the physical system may not work properly and hence put the safety of the human operators at risk. This is different from general-purpose systems where the usefulness drops in a more gradual manner (\eg a web service may tolerate a few millisecond delay without degrading user experience significantly). 

\begin{wrapfigure}{r}{0.455\textwidth}
%\begin{figure}[!ht]
%	\centering
	\hspace*{-1em}
	\includegraphics[scale=.32]{rt_node_new_arial.pdf}
	\caption{Abstraction of a real-time node with common uses cases.} 
	\label{fig:rts_struct}
%\end{figure}
\end{wrapfigure}

Figure~\ref{fig:rts_struct} presents a high-level illustration of a real-time CPS. Each real-time application in the system (called ``task'') represents a time-critical function and a collection of such tasks are hosted on a hardware platform. 
%The concept of tasks in RTS can be trivially mapped with \textit{processes} or \textit{threads} in general-purpose operating systems (OS).   
The scheduler in real-time operating system (RTOS) uses timers and interrupt handlers to enforce timing guarantees at runtime. This ability of the scheduler
to interrupt application processing at precise time instants
is essential to ensure the ``correctness'' of the system. Access to shared platform resources (such as caches, buses, memory) is regulated using resource sharing protocols to ensure data consistency and bounds on waiting time so that deadlines can be met.
The communication network in RTS is required to provide service with low
jitters and meet end-to-end message deadline for all messages.  

Although safety and fault-tolerance have long been important design
focus in such systems, security has rarely been a consideration in the design of RTS mainly due to beliefs such as: 
\ca RTS lack inherent value to adversaries (\textit{``why would anyone attack them?''}),
\cb the prevalence of custom hardware/software/protocols will deter attackers (\textit{``these protocols/hardware/software are secret and so arcane that no one can decipher them''}) and also 
\cc the lack of computing power and memory in these systems will throttle potential adversarial actions (\textit{``what can they do even if they get in?''}). While traditionally RTS adopted proprietary protocols, platforms, software and were air-gapped (\ie not connected to the outside world), with the advent of newer domains such as autonomous vehicles, drones, remote monitoring and control and Internet-of-Things (IoT)-specific applications, RTS find themselves front and center in modern society. Since many RTS now use commodity-off-the-shelf (COTS) components and are often connected to each other or even the Internet, they expose additional attack surfaces, often overturning all of the aforementioned beliefs. A number of high-profile attacks on real systems (\eg Stuxnet~\cite{stuxnet}, BlackEnergy~\cite{Ukraine16}), attack demonstrations by researchers on automobiles~\cite{ris_rts_1, checkoway2011comprehensive} and medical devices~\cite{security_medical} have shown that the threat is real and systems composed of RTS might be vulnerable to cyber attacks. 



Given the time and resource constraints under which RTS operate,
vulnerabilities in RTS differ considerably from those of traditional enterprise
systems. Threats faced by RTS could vary in scope and effect;
from the leakage of critical data \cite{embeddedsecurity:son2006} to hostile
actions due to lack of authentication
\cite{checkoway2011comprehensive, ris_rts_1, aircraft_hacking}. However, simply adding security
mechanisms that provide confidentiality (\eg encryption), integrity protection (\eg message authentication) and availability (\eg replication) without considering the real-time and embedded nature of such systems will not be effective. In the last few years there has been a lot of focus on securing critical CPS~\cite{xie2007improving,lin2009static,mohan_s3a,securecore,securecore_memory, securecore_syscal, kim2018securing,choi2018detecting, mhasan_resecure_iot}. A major focus of such work has been on securing communication protocols and on monitoring and detection mechanisms at the network and application level. Given the increasing cyber-attack risks, however, it is essential to have a layered
defense and integrate resilience against cyber attacks into the design of
controllers and actuators (\ie embedded RTS). It is also critical to retrofit existing controllers and actuators with protection, detection, survival and recovery mechanisms. In addition, stringent timing constraints severely inhibit how security solutions can be added to RTS; for instance, the \textit{protection methods should not cause timing problems} in RTS. 


\begin{mdframed}[backgroundcolor=gray!10]
	Real-time security mechanisms have to \textit{co-exist} with the existing  tasks in the system and have to operate \textit{without} impacting the timing and safety constraints of the control logic.\\ %[0.1em]
	\textbf{Challenge:} \em{how do we integrate security into RTS without impacting real-time guarantees?
	}
\end{mdframed}

\paragraph{Research Questions.}

%\todo{Complete}

As mentioned earlier, integrating security in RTS is not straightforward since monitoring/detection mechanisms must \ca co-execute with existing real-time tasks, \cb comply with timing/safety constraints, and \cc designed/scheduled in a way that an adversary cannot easily evade them. Security integration techniques for multicore RTS even more non-trivial in particular when considering that isolation in multicore systems is a challenge in itself. Our proposal focuses on addressing the following research questions.
\begin{itemize}[leftmargin=0.2in]
	\item How do we \textit{integrate} and then \textit{characterize the effects of security} in RTS especially those designed using \textit{multicore} chips?
	\item What are the \textit{trade-offs} on the security and timing requirements while guaranteeing \textit{no (or minimal) perturbations} for the real-time properties?
	\item What are the \textit{performance criteria and metrics} that need to be considered? %while integrating security into RTS?
\end{itemize}


%\todo{Complete}

\subsection{Proposed Research}

In this project we will develop an unified framework that will \textit{integrate monitoring and detection mechanisms into \textbf{multicore} RTS} (Section~\ref{sec:proposed_work}). Our proposed research will show that \textbf{\textit{it is possible to integrate security into RTS by a careful scheduler-level analysis of, and co-design with, system constraints, viz. software, hardware, and timing requirements}}. In answer to the above research questions, we will \textbf{\cab devise \textit{novel algorithms, scheduling models, and frameworks} that are cognizant with real-time requirements, \cbb build \textit{design-time tools and system-level plugins} to integrate our proposed techniques into off-the-shelf systems, and \ccb develop  \textit{metrics} to carefully trade-off two contending requirements: timeliness and security}. 
Our techniques will assist the designers of systems to better understand \ca how to \textit{integrate security} into RTS and \cb what are the \textit{trade-offs} on the security front while guaranteeing minimal (or no) perturbations for the real-time properties. Our end goal, then, is to \textit{provide designers with a knob that they can use to tune to one side or the other --- real-time vs. security}. 

We will evaluate our proposed techniques using \ca simulations, \cb realistic workloads and embedded benchmark suites (PapaBench~\cite{nemer2006papabench}, MiBench~\cite{guthaus2001mibench},  and  MultiBench~\cite{eembc_multibench}), and \cc off-the-shelf hardware testbeds
(ground rover~\cite{monsterborg} and robotic arm~\cite{robot_arm_rot3u})
[Section~\ref{sec:eval}].  Our research activities described in this proposal will be complemented with tightly integrated educational components reaching from undergraduate/graduate students, students from minority communities, and K-12 students [Section~\ref{sec:broader}]. 




%\paragraph{Research Questions.}



The main outcomes of our research will be to:

\begin{itemize}
	\itemsep 0pt
	\parskip 0pt
	\item Blah
	\item Blah; and
	\item Blah.
\end{itemize}



\paragraph{Intellectual Merit.}

\paragraph{Plans for Assessing Success.}




\subsection{Justification for Funding Request}


\paragraph{Foundations for Long-Term Research.} 

The activities proposed here are critical steps for the PI to launch his research career. The proposed work will provide solid foundations for developing security techniques in diverse domains --- from RTS to broader cyber-physical, IoT/edge systems, and even general-purpose computing platforms. While the immediate focus of this research is on integrating security monitoring mechanisms into multicore RTS, the PI believes that the ideas developed in this work can be extended to more general-purpose computing systems and will serve as the basis for future competitive research proposals (\eg NSF CAREER program). Once the proposed ideas have been conceptualized, in Year 3--5, the PI will study the security of hardware/software-based architectures such as real-time hypervisors~\cite{rtzvisor,rt_xen} and TrustZone-enabled RTS~\cite{mukherjee2019optimized,mhasan_iotsnp19}. In Year 5--7, the PI intends to study security/privacy issues of distributed cyber-physical systems such as UAV swarms~\cite{chmaj2015distributed} and vehicular networks~\cite{mhasan_v2x_survey_20}. %software-defined real-time networks~\cite{sdn_qos_rtss17, sdn_qos_infocom21}, and sensor-actuator networks~\cite{van1993sensor,lu2015real}. 
From Year 7 onward, the PI intends to investigate security and resiliency aspects of more general-purpose systems along with IoT/edge-style cyber-physical computing platforms and study emerging technologies such as smart manufacturing, autonomous/electronic vehicles, and robot-aided automation (especially for elderly/disable people). \textbf{Note:} we present the plans beyond Year 2 to show the potentials of our proposed research agenda and the PI's long-term goals.



%\paragraph{Nourishing Collaborative Efforts.}

\paragraph{Strengthening Collaborative Efforts.}

The PI is developing new collaborations with other researchers for interdisciplinary work. In his first nine months as a faculty, the PI has already initiated collaborations with Dr. Sergio Salinas (Wichita State), Dr. Gedare Bloom (U. of Colorado at Colorado Springs), and Dr. Shubhra Kanti (Auburn) on topics related to security and privacy of cyber-physical systems and human-robot interaction. We anticipate leading multiple joint NSF proposals in the future. These projects will not only solve complex, practical problems and  contribute to the growth and development of computing research but also allow the PI to establish a successful academic career.
%academic career. %and lead to joint NSF proposals in the future.

During his PhD, the PI has successfully collaborated with industrial research labs (SRI International and Toyota Motors) that result in multiple publications~\cite{mhasan_sri_17,mhasan_v2x_survey_20} and a patent~\cite{mhasan_sri_patent_1}. The PI intends to strengthen his ties with industrial research and contribute to improving the security of consumer products. Hence, there is a high likelihood that the PI's future research outcomes will be deployed and enhance the security and resiliency of critical cyber-physical systems.


\paragraph{Lack of Support for Seeding Future Research.} 

The PI devotes his career to building secure, trustworthy, and resilient cyber-physical computing platforms. An integral part of the PI's career goals is inspiring, educating, and mentoring the broader community including K-12, undergraduate, and graduate students, and fostering equity and diversity in STEM education. 
%The PI has successful track record of publicly releasing research implementations 
%(refer to the PI's GitHub repositories~\cite{mh_github}) 
% 
The PI intends to promote reproducibility in systems research and disseminate his scientific findings through open-source initiatives. 
 \textbf{The PI \ul{does not have sufficient funds}\footnote{The PI's startup funding supports only one graduate student for 24 months.} to support a doctoral student and initiate these research plans.  The funding support from NSF for the proposed research initiation activities will be one of the significant steps towards achieving the PI's long-term career goals.}

%\todo{talk about collaboration}

%\paragraph{Seeding Career Goals.}




%\subsection{PI's Expertise and Likelihood of Success}

\subsection{PI's Expertise}



The PI, with his diverse research background, is in a unique position to carry out the proposed
research agendas. The PI's research expertise includes \ca real-time and cyber-physical systems security~\cite{mhasan_rtss16,
	mhasan_ecrts17, mhasan_sri_17, mhasan_date18, mhasan_date20, mhasan_resecure_iccps,mhasan_resecure_iot,mhasan_iotsnp19}, \cb development of resilient cyber-physical networks~\cite{sdn_qos_rtss17,sdn_qos_secsdn20,sdn_qos_infocom21}, and \cc resource management in cellular wireless networks~\cite{mhasan_bc15,mhasan_tcom15_1,mhasan_tcom15_2,mhasan_twc14_2,mhasan_twc14_1}. 
%He brings extensive expertise in the design and analysis of resilient real-time cyber-physical systems. 
In recent years, the PI has been at the forefront of the research in real-time security and resiliency with contributions ranging from developing theoretical models~\cite{sdn_qos_infocom21,sdn_qos_rtss17,sdn_qos_secsdn20,mhasan_rtss16,
	mhasan_ecrts17, mhasan_date18, mhasan_date20} as well as designing system architectures~\cite{mhasan_resecure_iccps,mhasan_resecure_iot,mhasan_iotsnp19}. 
	His work on integrating security as a first-class principle of real-time schedulers~\cite{mhasan_rtss16} has won
the \textit{outstanding paper} and \textit{best student paper} awards at the IEEE RTSS. %\footnote{The premier conference in the field of real-time and embedded systems.} 
%The PI  brings extensive expertise in the design and analysis of resilient real-time cyber-physical systems. 
The PI has published his solutions in top real-time and networking conferences/journals including RTSS~\cite{mhasan_rtss16, sdn_qos_rtss17}, ECRTS~\cite{mhasan_ecrts17}, ICCPS~\cite{mhasan_resecure_iccps}, INFOCOM~\cite{sdn_qos_infocom21}, IoT~\cite{mhasan_resecure_iot}, TWC~\cite{mhasan_twc14_1,mhasan_twc14_2}, and Network~\cite{mhasan_network15}. 
%As detailed in Section~\ref{sec:reorder}, the PI has also recently obtained preliminary results, in particular, on developing randomization techniques for single-core real-time systems. 
%The PI also has access to necessary resources (\eg undergraduate/graduate students, hardware, testbed, and research facilities) at Wichita State University to carry out the proposed research agenda (Section~\ref{sec:eval}). 
The \textit{PI has all the resources} (\eg research computers, experimental platforms, and lab facilities) needed to complete this project successfully. 
The PI also has experience working with \textit{underrepresented communities} in computing, most notably \textit{women students} at all levels ---  doctoral, masters, undergraduate, and even middle school students~\cite{mhasan_middleschool_news}. Further, the PI will leverage Wichita State's existing outreach programs 
%\footnote{Supporting letters are attached.} 
(\eg Shocker Engineering Academy~\cite{wsu_sea} and Hub for Cybersecurity Education and Awareness~\cite{wsu_hcea}) as well as resources from other organizations (\eg CRA's undergraduate mentoring programs~\cite{cra_dreu} and NSA's K-12 initiatives~\cite{k12_cybertalk}) and work on broadening participation of high school students and minority communities in engineering.  Hence, the PI is ideal for developing real-time security techniques proposed in this project and fostering diversity in STEM education.




