


\subsection{Justification for Funding Request}


\paragraph{Foundations for Long-Term Research.}

\paragraph{Lack of Research Initiation Support.}

\todo{talk about collaboration}

\paragraph{Seeding Career Goals.}

%\subsection{PI's Expertise and Likelihood of Success}

\subsection{PI's Expertise}



The PI, with his diverse research background, is in a unique position to carry out the proposed
research agendas. The PI's research expertise includes \ca real-time cyber-physical systems security~\cite{mhasan_rtss16,
	mhasan_ecrts17, mhasan_sri_17, mhasan_date18, mhasan_date20, mhasan_resecure_iccps,mhasan_resecure_iot,mhasan_iotsnp19}, \cb development of resilient cyber-physical networks~\cite{sdn_qos_rtss17,sdn_qos_secsdn20,sdn_qos_infocom21}, and \cc resource management in cellular wireless networks~\cite{mhasan_bc15,mhasan_tcom15_1,mhasan_tcom15_2,mhasan_twc14_2,mhasan_twc14_1}. 
%He brings extensive expertise in the design and analysis of resilient real-time cyber-physical systems. 
In recent years, the PI has been at the forefront of the research in real-time security and resiliency with contributions ranging from developing theoretical models~\cite{sdn_qos_infocom21,sdn_qos_rtss17,sdn_qos_secsdn20,mhasan_rtss16,
	mhasan_ecrts17, mhasan_date18, mhasan_date20} as well as designing system architectures~\cite{mhasan_resecure_iccps,mhasan_resecure_iot,mhasan_iotsnp19}. 
	His work on integrating security as first-class principle of real-time schedulers~\cite{mhasan_rtss16} has won
the \textit{outstanding paper} and \textit{best student paper} awards at the IEEE RTSS. %\footnote{The premier conference in the field of real-time and embedded systems.} 
The PI 
brings extensive expertise in the design and analysis of resilient real-time cyber-physical systems. He has published his solutions in top real-time and networking conferences/journals including RTSS~\cite{mhasan_rtss16, sdn_qos_rtss17}, ECRTS~\cite{mhasan_ecrts17}, ICCPS~\cite{mhasan_resecure_iccps}, INFOCOM~\cite{sdn_qos_infocom21}, and IoT~\cite{mhasan_resecure_iot}. 
As detailed in Section~\ref{sec:reorder}, the PI has also recently obtained preliminary results, in particular, on developing randomization techniques for single-core real-time systems. 
%The PI also has access to necessary resources (\eg undergraduate/graduate students, hardware, testbed, and research facilities) at Wichita State University to carry out the proposed research agenda (Section~\ref{sec:eval}). 
The \textit{PI has all the resources} (\eg research computers, experimental platforms, and lab facilities) needed to complete this project successfully. 
The PI also has experience working with \textit{underrepresented communities} in computing, most notably \textit{women students} at all levels ---  doctoral, masters, undergraduate, and even middle school students. Further, the PI will leverage Wichita State's existing outreach programs 
%\footnote{Supporting letters are attached.} 
(\eg Shocker Engineering Academy~\cite{wsu_sea}, undergraduate research projects~\cite{wsu_netcps_reu}, and Hub for Cybersecurity Education and Awareness~\cite{wsu_hcea}) as well as resources form other organizations (\eg CRA's undergraduate mentoring programs~\cite{cra_dreu} and NSA's K-12 initiatives~\cite{k12_cybertalk}) and work on broadening participation of high school students and minority communities in engineering.  Hence, the PI is ideal for developing real-time security techniques such as those proposed in this project and fostering diversity in STEM education.


