
% reset section counter
\setcounter{section}{0}
%\renewcommand*{\theHsection}{\the\value{section}}. % uncomment if hyperref is enabled



\section{Types of Data}


We will not use any data from other sources except those generated from our research activities. We expect the following data to be generated as a direct result of the work being proposed in this project: 
\vspace*{0.5em}
\begin{itemize}
	\item Algorithms, design documents, and code for proposed schemes.
	\item Metrics to analyze the effectiveness of security integration techniques and the evaluated values based on these metrics.
	\item Use cases and test harnesses for our implementations.
	\item Source code and documentation related to integrating security techniques in commodity operating systems (RT\_PREEMPT, LITMUS\textsuperscript{RT}, and OP-TEE).
	\item Analyses and evaluation tools such as simulation frameworks and the design of hardware testbeds (ground rover and robotic arm).
	\item Design of experiments, experimental results, analysis, and videos for demonstrations.
%	\item Reference designs for CSAW and F1TENTH competitions developed by undergraduate/graduate students.
	\item Reference designs for URCAF competition developed by undergraduate students.
	\item Student reports, project designs, software artifacts, and theses.
	\item Scientific publications with research findings.
	\item Course materials (\eg lecture notes, presentation slides, exams, and project assignments).
\end{itemize}




\vspace*{1em}
\section{Data and Metadata Standards}


All data will be electronic. We will utilize industry standards and publicly accepted data formats. The descriptions, algorithms, theories, results will be documented in conference and journal papers, technical reports, research notes, and online files.  Source files will be stored in appropriate formats for the specific applications. 
%For example, most of the operating system implementations will be developed in the C language and will be stored in \texttt{.c} and \texttt{.h} files. 
We will document source code in standard ways that indicate distribution license as well as source information. Data sets will be retained in their original form (binary, CSV, XML, JSON). Videos for demonstrations will be recorded in standard format (MP4). If the formats are not standardized, we will provide the relevant documentation of the data formats.

\vspace*{1em}
\section{Policies for Access and Sharing and Provisions for Appropriate Protection/Privacy}

The PI retains the right to use all data generated as part of this research and anticipates sharing of data through publications and open source releases. There will be no copyright or licensing issues associated with the data being collected and maintained. This study will only collect non-sensitive data. We will not collect any private data in this research and no personal identifiers will be recorded or retained by the PI or students. Hence, there will be no privacy concerns.

Any vulnerabilities in existing systems, if discovered, will be dealt with in the following manner: \ca we will actively work with the vendors to develop patches/safeguards that can mitigate the issues; \cb we will further provide 3--6 months of lead times to vendors in case they wish to work on the fix themselves.
The PI will only publish the results once these processes have been completed or lead times have lapsed.

\vspace*{1em}
\section{Policies and Provisions for Re-use, Re-distribution}

We will not enforce any permissions or restrictions on the released code and data as we hope to foster the open access initiatives. We will use popular open-source licenses such as Apache, BSD, GPL, and MIT for our released code. We will choose the specific license on a case-by-case basis depending on whether other open-source projects are used in the development.

\vspace*{1em}
\section{Plans for Archiving and Preservation of Access}

We will store initial code and data on computers/testbed machines where they are generated. We will periodically back up the data using versioning infrastructure (Git). Data will be archived on local servers/versioning websites (GitHub) and managed for long-term storage. 
%Our data will be encrypted with enterprise-grade data protection software. 
We will preserve the code and data and open access to them for at least five years after the award. After five years, we will continue to preserve the code and data through regular backups and maintenance on our lab's computers/local storage servers as well as on a public archiving website (GitHub).

\vspace*{1em}
\section{Policies for Dissemination and Sharing of Research Results}



%\section{Reproducibility, Access, and Sharing}
%\section{Dissemination/Sharing of Data}

%\paragraph{Reproducibility.}



%\paragraph{Reproducibility and Open Source Release.} 

The PI is attentive to the broad dissemination of his research findings. By preserving and making project data accessible, we will inform the broader community and foster discovery and collaboration. Several source code and findings from the PI's prior research efforts are publicly available on GitHub (\url{https://github.com/mnwrhsn}). The PI currently maintains a public GitHub repository for his research group (\url{https: //github.com/CPS2RL}). We will also create a dedicated website for this project under our departmental domain (\url{https://www.wichita.edu/academics/engineering/eecs/cps2rl/realtimesecurity}). We will provide contact information on the website to communicate with the PI directly. All data obtained from the research will be fully documented and publicly available on our project website and GitHub repositories. The documentation will describe our experimental settings, including how the experiments were conducted and how the data was measured. Our open-source releases will enable replication of the experiments by other researchers and practitioners. 

The PI and graduate students will internally participate in research seminars at the university to highlight the latest/preliminary results. We will broadly disseminate significant research findings through scholarly publications. For instance, algorithms, research methods, implementation details, and results will be disseminated via conference, journal, and workshop publications. Electronic manuscripts of publications will be available for public access on our project website. We will provide links to the appropriate publisher in case copyright issues prevent us from uploading the publications. Extended research results will be made available via technical reports on our project website and archived on our institutional repository (\url{https://soar.wichita.edu}). Videos recorded for demonstrations will be made available via public streaming sites such as YouTube, and access links will be provided on the project website.  
Course modules and lecture slides developed based on this work will be publicly available on our website. We will not release instructor resources (such as solutions to lab problems and homework assignments) on our website; however, they will be made available to educators upon request. 

All materials will contain an acknowledgment of NSF support as follows: \textit{``This material is based upon work supported by the National Science Foundation under Grant No. (NSF grant number).''} Additionally, we will include a disclaimer that \textit{``any opinions, findings, and conclusions or recommendations expressed in this material are those of the author(s) and do not necessarily reflect the views of the National Science Foundation.''} The PI will also work with the Wichita State University’s Office of Research to ensure that our activities conform to NSF's research dissemination and sharing policies/requirements.

